% Discussion
% Interpret your results, discuss implications, acknowledge limitations

\label{sec:discussion}

Our findings show that Wikipedia articles receive steady, organic attention even without likes, follower counts, or recommendation algorithms. The consistently high average daily pageviews for topics such as \textit{Artificial intelligence} and \textit{Climate change} suggest that certain subjects attract ongoing interest based on their relevance, not because they are artificially boosted by a platform. This supports the idea that Wikipedia’s model encourages users to seek information directly rather than through algorithmic feeds.

The day-by-day pageview spikes we observed also point to how Wikipedia responds to real-world events. When a topic becomes relevant in the news, its page often experiences a short burst of additional views. This pattern highlights that attention on Wikipedia is driven by public curiosity and information needs rather than competitive engagement metrics.

These results connect back to our motivating question by showing that meaningful engagement can occur on a platform without gamified incentives. Wikipedia’s design—with open access, transparent editing, and no reward system—still results in predictable patterns of public attention and information-seeking. This suggests that steady participation and visibility can emerge from user-driven interest rather than algorithmic promotion.

There are also limitations to our analysis. Pageviews measure reader attention, not editor activity or content quality. Because our data only reflect viewing behavior, we cannot draw conclusions about how editors collaborate, how conflicts are resolved, or how governance decisions are made. Future work could combine pageview data with revision histories or talk page analysis to provide a more complete picture of participation dynamics.
