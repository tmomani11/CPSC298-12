This project examines how Wikipedia functions as a large-scale online community without relying on common social media incentives such as likes, follower counts, or algorithmic recommendation systems. Unlike platforms that depend on gamified engagement, Wikipedia emphasizes openness, transparency, and user-driven information seeking. To explore how attention forms on such a platform, we analyze sustained public interest in several Wikipedia articles using pageview data from the Wikimedia Pageviews REST API. Our Python script collects daily pageviews over 30-day windows and summarizes each article’s total and average daily views, allowing us to compare levels of attention across topics.

The results show that Wikipedia articles receive steady, organic traffic even without algorithmic boosts. Topics such as \textit{Artificial intelligence} and \textit{Climate change} consistently record high average daily views, while other topics display occasional spikes linked to real-world events. These patterns highlight how public curiosity and information needs, rather than reward mechanisms, drive engagement on Wikipedia. Our findings support the idea that meaningful participation can emerge on platforms that prioritize transparent access and collaborative knowledge rather than competitive metrics. The study contributes a simple, data-driven look at how attention is distributed on a non-gamified platform and offers insights for designing healthier online spaces that are not dependent on algorithmic amplification.
