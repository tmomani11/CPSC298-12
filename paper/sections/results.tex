% Results
% Follows assignment: state operationalized RQ, focus on findings only

\label{sec:results}

This section answers our operationalized research question:
\emph{How much sustained public attention do selected Wikipedia articles receive over a 30-day period based on average daily pageviews?}
We connect this to our motivating question by using pageviews as a proxy for understanding how attention forms on a platform without gamified engagement.

Using our \texttt{week7.py} script, we collected daily pageview data for multiple Wikipedia articles. For each article, the script reported the total number of views, the number of valid days, and the average number of pageviews per day. These values were saved in \texttt{results\_week7.csv} and displayed in a comparison table.

Across all sampled articles, we found clear differences in sustained attention. Topics such as \textit{Artificial intelligence} and \textit{Climate change} showed the highest average daily pageviews, indicating that these articles consistently attracted more readers throughout the month. Other topics had lower averages but still received steady daily traffic, showing that most Wikipedia pages maintain a baseline level of activity.

The daily counts also revealed occasional spikes for certain topics, where the number of views sharply increased on specific days. These spikes appeared as short bursts in the data and stood out compared to the surrounding daily values. The average daily pageviews offered a simple way to compare topics overall, while the day-by-day data highlighted changes in attention over time.

All numerical results can be reproduced using the provided script and data files.
