% Introduction
% Structure: Problem/Motivation -> Background -> Research Questions -> Contributions -> Paper Outline

Wikipedia is one of the most active online communities in the world, yet it operates without many of the incentive structures found on traditional social media. There are no “likes,” follower counts, or algorithms recommending content. Instead, it relies on intrinsic motivation, transparent governance, and community-driven norms to sustain millions of contributors. This contrast raises an important question: how does Wikipedia succeed in maintaining large-scale participation and information quality without gamification?

This paper explores Wikipedia as a social media platform defined by collaboration rather than competition. We examine how editors coordinate, how conflicts are resolved through talk pages, and how governance mechanisms shape user behavior. Understanding this system provides insights into designing healthier online ecosystems.

\textbf{Research Questions:}
\begin{enumerate}
    \item How does the absence of likes, follower counts, and other social media metrics shape participation and motivation on Wikipedia?
    \item In what ways do talk pages, edit wars, and bursts of editing reflect both psychological conflict and community self-regulation?
    \item Which features of Wikipedia’s institutional design—such as open edit histories and transparent governance—enable rational collaboration despite conflict?
    \item Can these governance practices be adapted by other social platforms to foster more meaningful engagement and trust?
\end{enumerate}

\textbf{Contributions.} This paper provides:  
(1) a comparative analysis between Wikipedia and gamified platforms;  
(2) a data-driven examination of participation dynamics; and  
(3) implications for the design of transparent, trust-based online communities.

\textbf{Outline.} Section~\ref{sec:relatedwork} reviews prior research on online collaboration and motivation. Section~\ref{sec:methodology} describes our data collection and analysis methods. Section~\ref{sec:results} presents early findings, followed by discussion and conclusions.
