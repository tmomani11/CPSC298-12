% Introduction
% Structure: Problem/Motivation -> Background -> Research Questions -> Contributions -> Paper Outline

Wikipedia is one of the most active online communities in the world, yet it operates without many of the incentive structures found on traditional social media. There are no “likes,” follower counts, or recommendation algorithms guiding visibility. Instead, Wikipedia relies on intrinsic motivation, transparent governance, and community-driven norms to sustain millions of contributors. This raises an important question: how does Wikipedia maintain large-scale participation and engagement without gamification?

This paper approaches Wikipedia as a collaborative information platform rather than a competitive social network. Instead of focusing on social validation metrics, Wikipedia emphasizes openness, editing transparency, and community accountability. Understanding how engagement emerges under these conditions can offer insights for designing healthier online ecosystems.

\textbf{Motivating Research Question:}
\begin{itemize}
    \item How does Wikipedia sustain public engagement and information quality without the help of likes, follower counts, or algorithmic promotion?
\end{itemize}

Because this broad question cannot be directly measured, we focus on a more specific operationalized question that can be answered with available data:

\textbf{Operationalized Research Question:}
\begin{itemize}
    \item How much sustained public attention do selected Wikipedia articles receive over a 30-day period, based on their average daily pageviews?
\end{itemize}

\textbf{Contributions.}  
This paper provides:  
(1) a comparison of sustained attention across several Wikipedia articles using publicly available pageview data;  
(2) an examination of how patterns in attention reflect Wikipedia’s non-gamified model of engagement; and  
(3) implications for understanding how platforms can support meaningful participation without algorithmic incentives.

\textbf{Outline.}  
Section~\ref{sec:relatedwork} reviews prior research on online collaboration and participation.  
Section~\ref{sec:methodology} explains our data collection and proxy-based methodology.  
Section~\ref{sec:results} presents findings from the pageview analysis.  
Section~\ref{sec:discussion} discusses what these results suggest about engagement on Wikipedia, and Section~\ref{sec:conclusion} concludes.
