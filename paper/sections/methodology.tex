% Methodology
% Be specific enough that someone could reproduce your work

\label{sec:methodology}

\subsection{Overview}
Our main goal was to measure how much attention different Wikipedia articles receive over time. Since participation and motivation cannot be observed directly, we used pageview statistics as a simple proxy for public engagement. This allowed us to connect our broader research question about Wikipedia’s non-gamified structure to data we could actually collect.

\subsection{Data Collection}
We used the Wikimedia Pageviews REST API to gather daily pageview counts for several Wikipedia articles. Our Python script (\texttt{week7.py}) handles all API requests. By default, it collects the last 30 full days of data, but the script also supports custom date ranges.

For each article, the script retrieves:
\begin{itemize}
    \item The number of daily records returned,
    \item The total pageviews over the selected period, and
    \item The average number of pageviews per day.
\end{itemize}

These results are shown in a table when the script runs and also saved to a CSV file (\texttt{results\_week7.csv}) for later use.

\subsection{Data Processing}
Inside the script, the \texttt{summarize()} function takes the raw API output and calculates basic summary statistics. This includes the total views, the number of valid days, and the average daily views. All articles follow the same formatting, making it easier to compare them side-by-side.

\subsection{Proxy and Connection to Research Question}
Because we cannot directly measure things like editor motivation or user participation, we treat average daily pageviews as a proxy for public attention. This gives us a measurable way to compare how often different topics are accessed. Using this proxy helps connect our motivating research question to the dataset we were able to collect.

\subsection{Reproducibility}
All code and data are included in the project repository. Anyone can reproduce our results by running:

\begin{verbatim}
python week7.py "Artificial intelligence" "Climate change"
\end{verbatim}

This will regenerate the summary table and produce a new \texttt{results\_week7.csv} file with the exact values reported in this paper.
