% Related Work
% Organize by themes/categories, not paper-by-paper
% Show how your work builds on, differs from, or fills gaps in existing work
\label{sec:relatedwork}

Research on Wikipedia intersects with work on online collaboration, governance, and contributor motivation. Below we summarize two central themes supported by openly available studies.

\subsection{Wikipedia as a Social System}
Work by \citet{halfaker2013rise} frames Wikipedia as a dynamic social ecosystem whose technical controls shape patterns of participation. Their large-scale analysis of the English Wikipedia shows how quality-control mechanisms—such as automated edit filters and bureaucratic review processes—initially improved reliability but later discouraged newcomers and reduced overall editor diversity. The study highlights that Wikipedia’s governance model is inseparable from its social structure, where policies and technical tools jointly determine the health of the community.

\subsection{Motivation and Participation}
Complementary research by \citet{forte2013why} explores why individuals contribute voluntarily to Wikipedia despite the absence of gamified incentives. Through surveys and interviews, the authors find that intrinsic motives—altruism, knowledge sharing, and identification with the community—predict sustained engagement more strongly than external rewards. This perspective suggests that Wikipedia maintains participation through social capital and shared norms rather than points, badges, or follower counts.

Together, these findings portray Wikipedia as a hybrid social-technical platform whose longevity depends on balancing openness with governance and sustaining intrinsic contributor motivation in the absence of traditional gamification systems.
