% Related Work
% Organize by themes/categories, not paper-by-paper
% Show how your work builds on, differs from, or fills gaps in existing work

\label{sec:relatedwork}

Research on Wikipedia intersects with work on online collaboration, governance, and motivation. Below we summarize three central themes.

\subsection{Wikipedia as a Social System}
Early research established that Wikipedia functions as a large-scale collaborative system rather than a simple information repository. \citet{panciera2009wikipedians} described how a small core of dedicated editors perform most of the work, while \citet{halfaker2013riseanddecline} examined the “rise and decline” of editor participation due to increasing bureaucracy. These studies emphasize Wikipedia’s hybrid identity as both a social and technical platform.

\subsection{Motivation and Gamification}
Unlike social networks that use visible metrics such as karma or follower counts, Wikipedia’s design relies on intrinsic and social motivations. \citet{forte2013motivation} showed that editors are driven by altruism, expertise sharing, and a sense of community identity. Similarly, \citet{arazy2016quality} found that recognition within Wikipedia’s internal hierarchies replaces traditional gamification features, sustaining engagement without external rewards.

\subsection{Governance and Conflict Management}
Wikipedia’s open-editing model makes it a testing ground for studying digital governance. \citet{lam2011wpbias} identified systemic bias in participation, while \citet{viegas2007editwars} analyzed “edit wars” as natural but manageable elements of collaborative production. These works reveal that transparent edit histories, talk pages, and community norms help transform conflict into productive debate.

Together, these findings situate Wikipedia as a case study in decentralized, trust-based governance—an alternative model for digital platforms that seek to promote collective intelligence without competitive ranking systems.
