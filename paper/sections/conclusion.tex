% Conclusion
% Briefly summarize: problem, approach, findings, contributions
% End with a forward-looking statement

\label{sec:conclusion}

Wikipedia shows that large-scale online participation can thrive without likes, follower counts, or algorithmic recommendation systems. Instead of relying on competitive engagement metrics, Wikipedia attracts readers through openness, transparency, and user-driven information seeking. In this project, we examined how public attention forms on Wikipedia by analyzing 30 days of pageview data for several articles.

Our results showed steady, organic traffic across all sampled topics, with some pages—such as \textit{Artificial intelligence} and \textit{Climate change}—receiving consistently higher average daily views. We also observed occasional spikes in attention connected to real-world events, highlighting how Wikipedia serves as a go-to source when public interest increases. These findings support the idea that meaningful engagement can emerge on platforms that do not depend on gamified incentives.

This project contributes a simple, data-driven look at how attention is distributed on Wikipedia and offers insight into how non-gamified platforms can still maintain strong, sustained participation. Future work could combine pageview analysis with revision histories or editor activity data to better understand how attention interacts with contributions, collaboration patterns, and overall community health.
