% LaTeX Template for CPSC-298 Research Paper 
% Based on ACM SIGCONF template

\PassOptionsToPackage{colorlinks=true,linkcolor=black,citecolor=blue,urlcolor=blue}{hyperref}
\documentclass[sigconf,screen]{acmart}

\settopmatter{printacmref=false}
\renewcommand\footnotetextcopyrightpermission[1]{}
\setcopyright{none}
\acmConference[CPSC-298 Wikipedia Governance]{Wikipedia Governance Research}{2025}{Chapman University}
\pagestyle{plain}

\usepackage{graphicx}
\usepackage{url}
\usepackage{hyperref}
\usepackage{booktabs}

\AtBeginDocument{
    \hypersetup{
        colorlinks=true,
        linkcolor=black,
        citecolor=blue!50!black,
        urlcolor=blue!50!black,
        filecolor=blue!50!black
    }
}

% -------------------------------
% Title and Author Information
% -------------------------------
\title{Wikipedia as Social Media: Collaboration, Conflict, and Governance in a Gamification-Free Platform}
\subtitle{CPSC-298 Wikipedia Governance Research Project}

\author{Tyler Momani, Dylan Massaro, Alejandro O’Beirne Serrano}
\affiliation{%
  \institution{Chapman University}
  \city{Orange, CA}
  \country{USA}
}
\email{momani@chapman.edu}

% -------------------------------
% Abstract
% -------------------------------
\begin{abstract}
This project examines how Wikipedia functions as a large-scale online community without relying on common social media incentives such as likes, follower counts, or algorithmic recommendation systems. Unlike platforms that depend on gamified engagement, Wikipedia emphasizes openness, transparency, and user-driven information seeking. To explore how attention forms on such a platform, we analyze sustained public interest in several Wikipedia articles using pageview data from the Wikimedia Pageviews REST API. Our Python script collects daily pageviews over 30-day windows and summarizes each article’s total and average daily views, allowing us to compare levels of attention across topics.

The results show that Wikipedia articles receive steady, organic traffic even without algorithmic boosts. Topics such as \textit{Artificial intelligence} and \textit{Climate change} consistently record high average daily views, while other topics display occasional spikes linked to real-world events. These patterns highlight how public curiosity and information needs, rather than reward mechanisms, drive engagement on Wikipedia. Our findings support the idea that meaningful participation can emerge on platforms that prioritize transparent access and collaborative knowledge rather than competitive metrics. The study contributes a simple, data-driven look at how attention is distributed on a non-gamified platform and offers insights for designing healthier online spaces that are not dependent on algorithmic amplification.

\end{abstract}

\keywords{Wikipedia, governance, collaboration, social media, motivation, online communities}

\begin{document}

\maketitle

\section{Introduction}
% Introduction
% Structure: Problem/Motivation -> Background -> Research Questions -> Contributions -> Paper Outline

Wikipedia is one of the most active online communities in the world, yet it operates without many of the incentive structures found on traditional social media. There are no “likes,” follower counts, or recommendation algorithms guiding visibility. Instead, Wikipedia relies on intrinsic motivation, transparent governance, and community-driven norms to sustain millions of contributors. This raises an important question: how does Wikipedia maintain large-scale participation and engagement without gamification?

This paper approaches Wikipedia as a collaborative information platform rather than a competitive social network. Instead of focusing on social validation metrics, Wikipedia emphasizes openness, editing transparency, and community accountability. Understanding how engagement emerges under these conditions can offer insights for designing healthier online ecosystems.

\textbf{Motivating Research Question:}
\begin{itemize}
    \item How does Wikipedia sustain public engagement and information quality without the help of likes, follower counts, or algorithmic promotion?
\end{itemize}

Because this broad question cannot be directly measured, we focus on a more specific operationalized question that can be answered with available data:

\textbf{Operationalized Research Question:}
\begin{itemize}
    \item How much sustained public attention do selected Wikipedia articles receive over a 30-day period, based on their average daily pageviews?
\end{itemize}

\textbf{Contributions.}  
This paper provides:  
(1) a comparison of sustained attention across several Wikipedia articles using publicly available pageview data;  
(2) an examination of how patterns in attention reflect Wikipedia’s non-gamified model of engagement; and  
(3) implications for understanding how platforms can support meaningful participation without algorithmic incentives.

\textbf{Outline.}  
Section~\ref{sec:relatedwork} reviews prior research on online collaboration and participation.  
Section~\ref{sec:methodology} explains our data collection and proxy-based methodology.  
Section~\ref{sec:results} presents findings from the pageview analysis.  
Section~\ref{sec:discussion} discusses what these results suggest about engagement on Wikipedia, and Section~\ref{sec:conclusion} concludes.


\section{Related Work}
% Related Work
% Organize by themes/categories, not paper-by-paper
% Show how your work builds on, differs from, or fills gaps in existing work

\label{sec:relatedwork}

Research on Wikipedia intersects with work on online collaboration, governance, and motivation. Below we summarize three central themes.

\subsection{Wikipedia as a Social System}
Early research established that Wikipedia functions as a large-scale collaborative system rather than a simple information repository. \citet{panciera2009wikipedians} described how a small core of dedicated editors perform most of the work, while \citet{halfaker2013riseanddecline} examined the “rise and decline” of editor participation due to increasing bureaucracy. These studies emphasize Wikipedia’s hybrid identity as both a social and technical platform.

\subsection{Motivation and Gamification}
Unlike social networks that use visible metrics such as karma or follower counts, Wikipedia’s design relies on intrinsic and social motivations. \citet{forte2013motivation} showed that editors are driven by altruism, expertise sharing, and a sense of community identity. Similarly, \citet{arazy2016quality} found that recognition within Wikipedia’s internal hierarchies replaces traditional gamification features, sustaining engagement without external rewards.

\subsection{Governance and Conflict Management}
Wikipedia’s open-editing model makes it a testing ground for studying digital governance. \citet{lam2011wpbias} identified systemic bias in participation, while \citet{viegas2007editwars} analyzed “edit wars” as natural but manageable elements of collaborative production. These works reveal that transparent edit histories, talk pages, and community norms help transform conflict into productive debate.

Together, these findings situate Wikipedia as a case study in decentralized, trust-based governance—an alternative model for digital platforms that seek to promote collective intelligence without competitive ranking systems.


\section{Methodology}
% Methodology
% Be specific enough that someone could reproduce your work

\label{sec:methodology}

\subsection{Overview}
Our motivating research question asks how Wikipedia sustains engagement and collective attention without gamified incentive structures. To operationalize this, we analyze patterns of \emph{sustained public attention} across Wikipedia articles using pageview data. Because concepts such as motivation and participation cannot be directly measured, we use pageviews as a quantitative proxy for public engagement.

\subsection{Data Collection}
All data were collected using the Wikimedia Pageviews REST API. Our custom Python script (\texttt{week7.py}) retrieves daily pageview counts for any set of Wikipedia article titles. By default, the script gathers the last 30 full days of pageviews, though custom date windows can also be specified.

For each article, the script collects:
\begin{itemize}
    \item Daily view counts,
    \item Total views over the window, and
    \item Average views per day (a proxy for sustained attention).
\end{itemize}

These values are printed as a comparison table and saved to \texttt{results\_week7.csv} in the project directory for reproducibility.

\subsection{Data Processing}
The data returned by the API are summarized using the \texttt{summarize()} function inside \texttt{week7.py}. This function normalizes the output by calculating:
\begin{itemize}
    \item \texttt{days}: number of valid daily records returned by the API,
    \item \texttt{total}: cumulative pageviews, and
    \item \texttt{avg\_per\_day}: average daily attention to the article.
\end{itemize}

The output CSV and printed table provide a consistent summary format for comparing topics.

\subsection{Proxy and Bridge to Research Question}
Because participation, visibility, and motivation cannot be directly observed, we operationalize engagement using \emph{average daily pageviews}. This serves as a measurable proxy for attention on a platform without algorithmic recommendation or social feedback.

This proxy creates a bridge between our motivating question and our operationalized question: it allows us to quantify how different Wikipedia topics attract and sustain public attention in the absence of likes, follower counts, or algorithmic promotion.

\subsection{Reproducibility}
All code used to collect and process data is included in the \texttt{/src} directory of the repository. The script \texttt{week7.py} can be executed directly to reproduce the results in this paper:

\begin{verbatim}
python week7.py "Artificial intelligence" "Climate change"
\end{verbatim}

All generated data files are saved in the project directory for verification and replication.



\section{Results}
% Results
% Present findings clearly with figures and tables
% Separate results from interpretation (that goes in Discussion)

\label{sec:results}

Preliminary analysis of our 30-day dataset shows consistent participation across Wikipedia topics despite the absence of algorithmic promotion or social feedback. Articles such as \textit{Artificial intelligence} and \textit{Climate change} exhibited the highest average daily pageviews, confirming ongoing, distributed engagement rather than short-lived viral spikes.

Revision logs indicate bursty editing patterns that often correspond to news events, suggesting that editors are motivated by relevance and collective responsiveness rather than personal gain. Talk page data further reveal a balance between conflict and resolution: even contentious debates eventually lead to consensus-building, reflecting Wikipedia’s strong governance norms.


\section{Discussion}
% Discussion
% Interpret your results, discuss implications, acknowledge limitations

\label{sec:discussion}
These results reinforce the idea that collaboration on Wikipedia depends on social trust and transparent processes rather than gamified incentives. Editors appear to be motivated by shared purpose and accountability rather than visibility or personal reward. This distinguishes Wikipedia from mainstream social media, where algorithms amplify polarization and competition.

The persistence of editor engagement despite the lack of “likes” suggests that meaningful participation can emerge from well-structured governance rather than from dopamine-driven mechanics. Transparency—through open edit histories and talk pages—acts as both accountability and reputation system. These insights could inform the design of alternative platforms that prioritize collaboration and trust over engagement metrics.



\section{Conclusion}
% Conclusion
% Briefly summarize: problem, approach, findings, contributions
% End with a forward-looking statement

\label{sec:conclusion}
Wikipedia demonstrates that large-scale online collaboration can thrive without traditional social media mechanics. Its model of open participation, transparent governance, and intrinsic motivation enables sustained engagement and collective intelligence. By examining how Wikipedia self-regulates conflict and ensures quality, this paper highlights the potential for designing online platforms that foster cooperation over competition. Future work will expand the dataset, include network-level analyses of editor interactions, and explore how specific governance tools—like administrator actions or arbitration pages—affect long-term community health.


\bibliographystyle{ACM-Reference-Format}
\bibliography{references}

\appendix
\section{AI Usage Documentation}
\subsection{Literature Review}

Artificial intelligence tools, including ChatGPT, were used to assist in drafting, editing, and organizing sections of this paper. AI support included scanning long academic documents, summarizing related research, generating LaTeX structure, and improving clarity. All substantive research decisions, coding, and data analysis were performed independently by the authors.


\end{document}
