% LaTeX Template for CPSC-298 Research Paper 
% Based on ACM SIGCONF template

\PassOptionsToPackage{colorlinks=true,linkcolor=black,citecolor=blue,urlcolor=blue}{hyperref}
\documentclass[sigconf,screen]{acmart}

\settopmatter{printacmref=false}
\renewcommand\footnotetextcopyrightpermission[1]{}
\setcopyright{none}
\acmConference[CPSC-298 Wikipedia Governance]{Wikipedia Governance Research}{2025}{Chapman University}
\pagestyle{plain}

\usepackage{graphicx}
\usepackage{url}
\usepackage{hyperref}
\usepackage{booktabs}

\AtBeginDocument{
    \hypersetup{
        colorlinks=true,
        linkcolor=black,
        citecolor=blue!50!black,
        urlcolor=blue!50!black,
        filecolor=blue!50!black
    }
}

% -------------------------------
% Title and Author Information
% -------------------------------
\title{Wikipedia as Social Media: Collaboration, Conflict, and Governance in a Gamification-Free Platform}
\subtitle{CPSC-298 Wikipedia Governance Research Project}

\author{Tyler Momani, Dylan Massaro, Alejandro O’Beirne Serrano}
\affiliation{%
  \institution{Chapman University}
  \city{Orange, CA}
  \country{USA}
}
\email{momani@chapman.edu}

% -------------------------------
% Abstract
% -------------------------------
\begin{abstract}
% Abstract: 150-250 words summarizing your entire paper
% Include: problem, methods, key findings, contribution

This project explores Wikipedia as a unique form of social media that thrives without conventional mechanisms of online validation such as likes, follower counts, or karma. Despite its minimalist interface and lack of gamified features, Wikipedia has sustained one of the largest and most productive online communities in history. This study investigates how collaboration, conflict, and governance function within this nontraditional platform. Using data from the Wikipedia API—such as revision histories, talk page interactions, and pageview statistics—we analyze how contributors engage, self-regulate, and maintain information quality. By comparing these patterns with social platforms like Reddit and Twitter, we identify how Wikipedia’s open, transparent, and accountability-driven design fosters participation without the toxicity often associated with algorithmic engagement. Ultimately, the paper argues that Wikipedia’s governance model offers valuable lessons for designing healthier digital spaces that balance openness, trust, and collective responsibility.



\end{abstract}

\keywords{Wikipedia, governance, collaboration, social media, motivation, online communities}

\begin{document}

\maketitle

\section{Introduction}
% Introduction
% Structure: Problem/Motivation -> Background -> Research Questions -> Contributions -> Paper Outline

Wikipedia is one of the most active online communities in the world, yet it operates without many of the incentive structures found on traditional social media. There are no “likes,” follower counts, or algorithms recommending content. Instead, it relies on intrinsic motivation, transparent governance, and community-driven norms to sustain millions of contributors. This contrast raises an important question: how does Wikipedia succeed in maintaining large-scale participation and information quality without gamification?

This paper explores Wikipedia as a social media platform defined by collaboration rather than competition. We examine how editors coordinate, how conflicts are resolved through talk pages, and how governance mechanisms shape user behavior. Understanding this system provides insights into designing healthier online ecosystems.

\textbf{Research Questions:}
\begin{enumerate}
    \item How does the absence of likes, follower counts, and other social media metrics shape participation and motivation on Wikipedia?
    \item In what ways do talk pages, edit wars, and bursts of editing reflect both psychological conflict and community self-regulation?
    \item Which features of Wikipedia’s institutional design—such as open edit histories and transparent governance—enable rational collaboration despite conflict?
    \item Can these governance practices be adapted by other social platforms to foster more meaningful engagement and trust?
\end{enumerate}

\textbf{Contributions.} This paper provides:  
(1) a comparative analysis between Wikipedia and gamified platforms;  
(2) a data-driven examination of participation dynamics; and  
(3) implications for the design of transparent, trust-based online communities.

\textbf{Outline.} Section~\ref{sec:relatedwork} reviews prior research on online collaboration and motivation. Section~\ref{sec:methodology} describes our data collection and analysis methods. Section~\ref{sec:results} presents early findings, followed by discussion and conclusions.


\section{Related Work}
% Related Work
% Organize by themes/categories, not paper-by-paper
% Show how your work builds on, differs from, or fills gaps in existing work

\label{sec:relatedwork}

Research on Wikipedia intersects with work on online collaboration, governance, and motivation. Below we summarize three central themes.

\subsection{Wikipedia as a Social System}
Early research established that Wikipedia functions as a large-scale collaborative system rather than a simple information repository. \citet{panciera2009wikipedians} described how a small core of dedicated editors perform most of the work, while \citet{halfaker2013riseanddecline} examined the “rise and decline” of editor participation due to increasing bureaucracy. These studies emphasize Wikipedia’s hybrid identity as both a social and technical platform.

\subsection{Motivation and Gamification}
Unlike social networks that use visible metrics such as karma or follower counts, Wikipedia’s design relies on intrinsic and social motivations. \citet{forte2013motivation} showed that editors are driven by altruism, expertise sharing, and a sense of community identity. Similarly, \citet{arazy2016quality} found that recognition within Wikipedia’s internal hierarchies replaces traditional gamification features, sustaining engagement without external rewards.

\subsection{Governance and Conflict Management}
Wikipedia’s open-editing model makes it a testing ground for studying digital governance. \citet{lam2011wpbias} identified systemic bias in participation, while \citet{viegas2007editwars} analyzed “edit wars” as natural but manageable elements of collaborative production. These works reveal that transparent edit histories, talk pages, and community norms help transform conflict into productive debate.

Together, these findings situate Wikipedia as a case study in decentralized, trust-based governance—an alternative model for digital platforms that seek to promote collective intelligence without competitive ranking systems.


\section{Methodology}
% Methodology
% Be specific enough that someone could reproduce your work


\label{sec:methodology}

\subsection{Data Collection}
Our study uses publicly available data from the Wikipedia API and the Wikimedia Pageviews REST API. We collected:
\begin{itemize}
    \item Revision histories to track editing patterns and bursts of activity,
    \item Talk page discussions to capture instances of negotiation and conflict, and
    \item Pageview statistics for measuring sustained public attention.
\end{itemize}

\subsection{Data Processing}
Using a custom Python script (\texttt{week7.py}), we queried the APIs to retrieve edit histories and daily pageviews for selected topics over 30-day periods. These results were stored in CSV format for reproducibility and comparative analysis.

\subsection{Comparative Analysis}
We compare Wikipedia’s participation dynamics with those observed on platforms such as Reddit and Twitter, where visible feedback metrics (upvotes, likes, retweets) heavily influence engagement. Our approach focuses on how Wikipedia maintains motivation and community regulation without these mechanisms.

\subsection{Rationale for Method Choices}
We chose quantitative pageview data to capture sustained user attention and qualitative analysis of talk pages to observe governance in action. This mixed-method approach highlights both the behavioral and structural factors enabling large-scale collaboration in a non-gamified environment.


\section{Results}
% Results
% Present findings clearly with figures and tables
% Separate results from interpretation (that goes in Discussion)

\label{sec:results}

Preliminary analysis of our 30-day dataset shows consistent participation across Wikipedia topics despite the absence of algorithmic promotion or social feedback. Articles such as \textit{Artificial intelligence} and \textit{Climate change} exhibited the highest average daily pageviews, confirming ongoing, distributed engagement rather than short-lived viral spikes.

Revision logs indicate bursty editing patterns that often correspond to news events, suggesting that editors are motivated by relevance and collective responsiveness rather than personal gain. Talk page data further reveal a balance between conflict and resolution: even contentious debates eventually lead to consensus-building, reflecting Wikipedia’s strong governance norms.


\section{Discussion}
% Discussion
% Interpret your results, discuss implications, acknowledge limitations

\label{sec:discussion}
These results reinforce the idea that collaboration on Wikipedia depends on social trust and transparent processes rather than gamified incentives. Editors appear to be motivated by shared purpose and accountability rather than visibility or personal reward. This distinguishes Wikipedia from mainstream social media, where algorithms amplify polarization and competition.

The persistence of editor engagement despite the lack of “likes” suggests that meaningful participation can emerge from well-structured governance rather than from dopamine-driven mechanics. Transparency—through open edit histories and talk pages—acts as both accountability and reputation system. These insights could inform the design of alternative platforms that prioritize collaboration and trust over engagement metrics.



\section{Conclusion}
% Conclusion
% Briefly summarize: problem, approach, findings, contributions
% End with a forward-looking statement

\label{sec:conclusion}
Wikipedia demonstrates that large-scale online collaboration can thrive without traditional social media mechanics. Its model of open participation, transparent governance, and intrinsic motivation enables sustained engagement and collective intelligence. By examining how Wikipedia self-regulates conflict and ensures quality, this paper highlights the potential for designing online platforms that foster cooperation over competition. Future work will expand the dataset, include network-level analyses of editor interactions, and explore how specific governance tools—like administrator actions or arbitration pages—affect long-term community health.


\bibliographystyle{ACM-Reference-Format}
\bibliography{references}

\appendix
\section{AI Usage Documentation}
\subsection{Literature Review}

Artificial intelligence tools, including ChatGPT, were used to assist in drafting, editing, and organizing sections of this paper. AI support included scanning long academic documents, summarizing related research, generating LaTeX structure, and improving clarity. All substantive research decisions, coding, and data analysis were performed independently by the authors.


\end{document}
